\documentclass[eng,oneside]{mgr}
\usepackage[polish]{babel}
\usepackage[utf8]{inputenc}
\usepackage{polski}
\usepackage[hidelinks]{hyperref}
\author{Marcin Mantke}
\title{System zarządzania inteligentnym domem z wykorzystaniem Raspberry Pi oraz technologii internetowych.}
\engtitle{Smart house management system using Raspberry Pi and Web technologies.}
\supervisor{dr inż. Marek Piasecki}
\field{Informatyka (INF)}
\specialisation{Inżynieria systemów informatycznych (INS)}
\date{2015}
\begin{document}
\maketitle
\tableofcontents
\chapter{Wprowadzenie}
\section{Cel i zakres}
\section{Przegląd istniejących rozwiązań}
\subsection{Domoticz.com}
Od czasu popularyzacji rozwiązań pokroju Arduino i Raspberry Pi, hobbystyczne projekty inteligentnych domów są coraz częściej realizowane. Sprzyja temu fakt, że ceny podzespołów wymaganych do realizacji projektu są coraz niższe, a osoby zainteresowane mają coraz więcej literatury dostępnej w Internecie. Takimi właśnie hobbystami byli twórcy platformy \textit{Domoticz}. Jest to zagraniczny serwis udostępniający multiplatformowe rozwiązania dla inteligentnych domów. Jest on skierowany głównie do hobbystów. Jak można przeczytać na stronie domowej projektu (\url{http://www.domoticz.com/}), \textit{Domoticz} jest systemem automatyki domowej, który pozwala na monitorowanie i konfigurację urządzeń, takich jak: światła, przełączniki, różnego rodzaju sensory i mierniki, jak np temperatury, deszczu, wiatru, UV, prądu, gazu i wody.

Serwis ten udostępnia biblioteki umożliwiające podłączenie sensorów oraz oprogramowanie jednostki bazowej systemu (zwykle Raspberry Pi). Jako, że udostępniane są biblioteki, a nie tylko gotowe moduły sprzętowe, całość jest bardziej elastyczna. Oczywiście są tu ograniczenia, zarówno hardware'owe, jak i software'owe, lecz są one mniejsze niż w przypadku gotowych rozwiązań.

Samo oprogramowanie jest darmowe (Licencja GNU), ze strony producenta możliwy jest zakup urządzeń współpracujących z jego systemem, jednak możliwe jest również tworzenie sensorów we własnym zakresie, przy użyciu posiadanych podzespołów oraz udostępnionych bibliotek.
\subsection{Fibaro}                                                                                              
Rozwiązania oferowane przez firmę \textit{Fibaro} mają odmienną filozofię od firmy \textit{Domoticz}. Firma ta jest ukierunkowana na rozwiązania komercyjne. Oferuje ona systemy automatyki budynkowej, składające się z gotowych podzespołów, które trzeba jedynie zainstalować w budynku oraz skonfigurować.



System Fibaro opiera się na sieci \textit{mesh}. Urządzenia łączą się ze sobą za pośrednictwem protokołu \textit{Z-wave}.
\section{Zarys koncepcji}
\section{Wybrane technologie}

\chapter{Projekt}
\section{Topologia systemu}
\url{https://pl.wikipedia.org/wiki/Topologia_gwiazdy}
\end{document}